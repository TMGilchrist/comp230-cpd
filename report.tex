% Please do not change the document class
\documentclass{scrartcl}

% Please do not change these packages
\usepackage[hidelinks]{hyperref}
\usepackage[none]{hyphenat}
\usepackage{setspace}
\doublespace %turn on after writing!

% You may add additional packages here
\usepackage{amsmath}

% Please include a clear, concise, and descriptive title
\title{Proverbs 1:5}

% Please do not change the subtitle
\subtitle{COMP230 - CPD Report}

% Please put your student number in the author field
\author{1604281}

\begin{document}

\maketitle

\section{Introduction}

Striving to achieve slightly more character development than a lobotomised Mary Sue fan-fiction protagonist, we delve once more into the realm of continuing personal development. While I have drawn upon my experience across all my modules this term in identifying key skills to improve, I have mostly found challenges in the group game project, and therefore many of the skills below relate to this. Even taking this into account, I believe all the skills I have identified can be useful across a variety of modules as well as in Industry.


\section{Interpersonal Domain}

Communication is an important part of the group game project and overall I feel the team including myself have done well with keeping everyone informed of what we are doing in an informal manner. However, during more formal meetings – sprint planning and sprint review – I don’t engage as much with the discussion as I could do. This is partly due to wanting to get on with work rather than talk about it, especially as meetings can sometimes become longer than intended. These sorts of meetings are obviously an important part of industry however - especially if there are multiple teams working semi-independently on different aspects of the project – and learning to engage in them would benefit me.

The sprint planning and sprint review meetings are also sometimes a little chaotic, possibly due to the team not having a lot of experience working in an Agile way. For example, tasks are sometimes extremely broad and would require multiple people to complete them while during sprint reviews, tasks are sometimes moved to complete without being reviewed by the team. Being more vocal and engaging in the meetings to point out these potential issues and suggest ways to solve them could help the development experience for the whole team.

\subsection{Action}

I will endeavour to bring up one constructive point of feedback for the meeting process each sprint.


\section{Procedural Domain}

In addition to this however, my own ability to break features of the project down into specific and granular tasks is somewhat lacking. As I have experienced in our own meetings, broad or vaguely defined tasks and sprint goals can lead to some confusion over how tasks can be accomplished, as well as less engagement in meetings. It is also an important skill in industry where deadlines and project goals are more crucial, as being able to plan the necessary tasks for a feature allows better estimation of completion time.

Currently, I feel that I have trouble locking down individual tasks for a feature of the game, preferring to work on the feature and note the tasks I accomplished to implement the feature afterwards. Although I do try and define granular tasks, I often end up needing to expand on them during development. While I think a large part of this is based on my relative lack of experience with industry-style development, it is also definitely a lack of discipline on my part – and as mentioned earlier – being content to allow vague tasks during sprint planning meetings. Focusing on creating tasks before doing them, and adding them to the project backlog before sprint planning meetings would make my own work much more organised. 


\subsection{Action}

When writing tasks, I will break them down into a task small enough that I am confident I can complete it within a day, even if each feature must be broken down into many tasks. Additionally, I will only start tasks that are in the sprint backlog.


\section{Dispositional Domain}

Closely related to the last skill is the ability to stay focused on one problem or task at a time. In a task-driven development cycle it is important to finish the tasks that you take on in order to make progress on the project as a whole. This is especially true when other team members may be relying on the completion of a specific task to work on another feature. While I strive to complete tasks, I often find it hard to resist helping others with different tasks or experimenting with suggested features. Because of this, I fairly often find myself making progress in several different areas of the project, but not fully finishing them when I intended to. 

This is a bad habit as it leads to lack of synchronisation between the tasks I am working on and the tasks I have selected from the project backlog, which reduces the usefulness of the scrum board as a communication tool. While helping team members is obviously still a good thing to do, I need to make sure that I don’t get distracted from my own work trying to fix problems immediately, instead of offering a few solutions and checking back after.


\subsection{Action} 

Each sprint, I will ensure that I work only on my chosen task from the sprint backlog until it is completed.


\section{Affective Domain}

While I have not encountered issues with my personal emotions and mental state during the group game project, the ability to self-reflect and analyse emotional and mental weaknesses as part of continuing personal development is a powerful skill that can help to improve on current issues. However, this is a skill that I am fairly weak in as I find it difficult to identify specific issues that have troubled me. This is partly due to me dealing with problems without getting stressed or worrying about them, which in turn makes it hard consider what challenges are concerning and should be reflected upon.

It may also be somewhat caused by a subconscious desire to avoid discussing personal development with others, as I am occasionally - by nature – quite a private person. However, I recognise the benefits of self-analyses and will continue to attempt them. Further practice at reflection upon my performance and assessing my application of skills will help me to become more comfortable with the process. 


\subsection{Action} 

In next term's CPD report, I will take notes about my mental state and personal emotions during my weekly reports.


\section{Cognitive Domain}

Regarding programming knowledge, one area that I feel lacking in is my understanding of memory management in C++. Efficiency and robust code are both important in professional development, and memory management in C++ can greatly affect both. Currently, although I do understand general principles such as the need to delete pointers at the end of a program, I do not feel that I have a thorough comprehension of it and my garbage collection is likely to have some flaws in it because of this. 
While I am starting to gain a better understanding of some aspects of memory management, such as using references instead of pointers when handling matrices to avoid unnecessary copying, my knowledge is still very much based on code examples and tutorials. Further reading specifically into the subject of memory management would be useful.


\subsection{Action} 

I will spend half an hour a week researching C++ memory management techniques. 


\section{Conclusion}

It has been a promising and enjoyable term, especially regarding the group game project which has been an interesting learning experience – working once again with new people – but also immense fun. These CPD reports have pushed me to reflect on my own abilities and work, which has proven beneficial by highlighting my areas of weakness. 


\bibliographystyle{ieeetran}
%\bibliography{references}

\end{document}